\section{Chemolumineszenz von Luminol -- Experiment}
author: Ailin Sigel, Jonas R. Stöckmann

\subsubsection*{Chemikalien}

Destilliertes Wasser wird in diesem Versuch als Lösungsmittel verwendet.
\begin{enumerate}

\item Stammlösung L1
\begin{itemize}
\item 1 g Luminol ($C_8H_7N_3O_2$, 177,16$\frac{g}{mol}$)
\item 50 ml 10\%ige Natronlauge ($NaOH$, 39,997$\frac{g}{mol}$)
\item auf 500 ml aufgefüllt
\end{itemize}

\item Stammlösung L2
\begin{itemize}
\item 15 g Kaliumhexacyanoferrat(III) ($K_3[Fe(CN)_6]$, 329,26 $\frac{g}{mol})$)
\item auf 500 ml aufgefüllt 
\end{itemize}
\end{enumerate}

\subsubsection*{Durchführung} 

\begin{enumerate}

\item Grundlösung G1
\begin{itemize}
\item 30 ml L1 
\item auf 250 ml aufgefüllt
\end{itemize}

\item Grundlösung G2
\begin{itemize}
\item 30 ml L2 
\item 2 ml 30\%iges Wasserstoffperoxid ($H_2O_2$, 34,02$\frac{g}{mol}$)
\item auf 250 ml aufgefüllt 
\end{itemize}
\end{enumerate}

Mit den Stammlösungen L1 und L2 werden die Grundlösungen G1 und G2 nach der obigen Zusammensetzung hergestellt. Es entstehen somit je 250 ml von G1 und G2. 
In einem abgedunkelten Raum werden beide Grundlösungen gleichzeitig in ein Becherglas zusammengegeben. 

\subsubsection*{Beobachtungen}
Beim Zusammengeben von G1 und G2 und danach emittiert die entstandene Lösung blaues Licht. Die Lumineszenz nimmt mit der Zeit ab, es findet ein Farbwechsel von Neongrün zu Blau und anschließend zu dunkelblau statt. Zudem erscheint die Lösung nach Einschalten des Lichts klar. 

\subsubsection*{Hypothesen}
\begin{itemize}
\item Energie wird in Form von Licht freigesetzt
\item es findet eine Reaktion statt
\item Farbwechsel durch Sinken der Intensität
\item farbiger Anteil von G2 reagiert ab
\end{itemize}

Um die Hypothesen zu bestätigen oder zu widerlegen, werden acht Versuche durchgeführt, die sich unter anderem mit der Veränderung der Temperatur und der Konzentration der Edukte, sowie mit der Abnahme der Intensität mit der Zeit und der Rückreaktion beschäftigen.

Als Kamera wurde eine Canon70D verwendet mit einem ISO von $1000$, einer Blende von $4.0$ und einer Belichtungsdauer von $\frac{1}{20}$ s. Es wurden Bilder im Abstand von ca. 2 Sekunden gemacht. Das Becherglas und die Kamera hatten einen Abstand von ca. 46,5 cm.
%K_3 [Fe(CN)_6] 
