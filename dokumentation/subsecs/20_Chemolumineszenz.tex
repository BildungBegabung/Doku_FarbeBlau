\section{Chemolumineszenz}
\authors{Jonas R. Stöckmann, Atousa Seyedian}

Bei der Chemolumineszenz handelt es sich um einen Prozess, der zur Lumineszenz führt. Hier sorgt ein Übergang in den elektronischen Grundzustand für die Emission von Licht. Im Gegensatz zu der Fluoreszenz oder Phosphoreszenz wird die Energie nicht durch externen Lichteinfall einer Lichtquelle aufgebracht, sondern durch eine chemische Reaktion.
Bei der Reaktion der Edukte entsteht das Produkt im angeregten Zustand. Zum Ende der Chemolumineszenz-Reaktion fällt das Produkt in den energetischen Grundzustand. 

\subsection{Das Licht der Chemolumineszenz}

Die Dauer des Leuchtens bei einer Chemolumineszenz ist vergleichbar mit der von Fluoreszenz. 
Die emittierten Lichtwellen befinden sich meist im sichtbaren Bereich des Spektrums und weisen Wellenlängen zwischen 400 und 700 nm auf. Aber auch in den Wellenlängenbereichen der Farben Ultraviolett und Infrarot kann nicht-sichtbares Licht emittiert werden. \cite{Doerfler}

\subsection{Energie der Chemolumineszenz}

Bei einer Chemolumineszenz wird hauptsächlich chemische Energie in Form von Licht und nicht nur (wie oftmals bei chemischen Reaktionen) in Form von thermischer Schwingungsenergie emittiert. Dabei wird vorausgesetzt, dass diese Energiefreisetzung auf einmal, das heißt nicht zwangsweise in mehreren Stufen erfolgt. Chandross und Sonntags Postulat von 1964 zum Thema Chemolumineszenz \cite{Chandross} besagt, dass die Reaktionsenthalpie (Energieumsetzung) in einem einzigen Reaktionsschritt, in geringer Zeit und in möglichst kleinem Volumen freigesetzt werden sollte.
Dennoch ist es möglich, dass ein Teil der Energie auch durch Schwingung abgegeben wird. 

\subsection{Ablauf der Reaktion}

Eines der bekanntesten Beispiele für Chemolumineszenz-Reaktionen ist die durch Kaliumhexacyanoferrat(III) katalysierte Oxidation von Luminol mit alkalischem Wasserstoffperoxid. 

\begin{dsafigure}
 \centering
 \includegraphics[width=\columnwidth]{Luminol_chemiluminescence_wiki.png}
 \caption{Chemoluminszenz-Reaktion von Luminol. \cite{Luminol_wiki}}
 \label{dsafigure:Chemolumineszenz}
\end{dsafigure}

Das Luminol reagiert in alkalischem Milieu durch Angriff von Hydroxid-Anionen zum Dianion. Das Luminol agiert dabei als Broensted-Säure und reagiert mit den $ OH^{-} $-Ionen unter Abspaltung von $ H^{+} $-Ionen. Das so entstandene Dianion ist mesomeriestabilisiert. 
Bei der folgenden Oxidationsreaktion des Dianions mit Wasserstoffperoxid wird Stickstoff freigesetzt. Es entsteht ein angeregtes Zwischenprodukt, das 3-Amino-Phtalsäure-dianion. Dieses instabile Intermediat geht zuletzt unter Lichtemission in den energetischen Grundzustand über.  
