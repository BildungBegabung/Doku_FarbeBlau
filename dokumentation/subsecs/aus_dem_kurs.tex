\section{Unser Kurs}
Wie ist ein Atom aufgebaut? Was ist Licht? Wie wechselwirken Teilchen? Wie entstehen Moleküle? Oder was sind Orbitale?
Das sind nur ein paar der Fragen, mit denen wir uns im Laufe unseres Kurses beschäftigt haben. Zuerst geschockt von der Komplexität der Quantenmechanik, stellten wir schnell fest, dass dieses Wissen elementar ist, um die spektroskopischen Eigenschaft von Stoffen zu erklären. Also auf deutsch: Warum ist ein Stoff farbig? In einer perfekt abgestimmten Mischung aus Theorie und Experimenten wurde blau zu unserer absoluten Lieblingsfarbe. Nicht nur weil wir eine blau leuchtende Flüssigkeit gemischt haben oder die Farben Berliner Blau und Ultramarin herstellten, sondern auch, weil wir nun in der Lage sind, mithilfe Berechnungen die Farbe eines Stoffes vorherzusagen. Dazu stellten wir die berühmte und gefürchtete Schrödingergleichung auf und konnten angeben, in welchem Bereich sich ein Elektron um den Atomkern aufhalten könnte und welche Energie es hat. Dabei bedienten wir uns eines wissenschaftlichen Hochleistungsprogrammes. Besonders gefallen hat allen die eigenständige Arbeit. Vor Beginn der Akademie bekamen wir ein Referatsthema zugeteilt, über das es sich zu informieren galt. Der Kurs gab uns die Möglichkeit auch unsere Erfahrungen im Vortragen zu verbessern. So lernten wir neben all den naturwissenschaftlichen Fakten auch, wie man eine Präsentation anschaulich gestaltet oder wie man ein konstruktives Feedback gibt. Nach jedem Vortrag "reflektierten" wir gemeinsam das Gehörte und versuchten dieses zu "absorbieren". Außerdem führten uns unsere Kursleiter an das wissenschaftliche Arbeiten heran. Denn das genaue Protokollieren von Versuchen und das Verfassen von wissenschaftlichen Texten muss gelernt sein und ist eine perfekte Vorbereitung auf das Studium, auf das wir nun alle sehr gespannt sind. Allerdings werden wir das morgendliche Singen unseres Elemente-Songs vermissen, das Musthave für jeden hardcore Chemiker:
There's ...