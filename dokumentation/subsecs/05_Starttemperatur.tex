\subsection{Temperaturabhängigkeit}
\authors{Johannes Wörsdörfer}
\subsubsection{Durchführung}
Für diesen Versuch werden die Gemische G1 und G2 auf 10$^\circ$ C abgekühlt. Anschließend werden sie, wie im allgemeinen Teil beschrieben, gemischt. Als Referenzwert werden beim zweiten Versuch die Gemische nicht abgekühlt, um eine Starttemperatur von ca. 20 $^\circ$C (Raumtemperatur) zu erreichen.

\subsubsection{Beobachtung}
Bei einer geringen Starttemperatur von ca. 10 $^\circ$C lässt sich eine schwächere Intensität des Lichtes als bei einer Starttemperatur von 20 $^\circ$C messen (siehe Tabelle \ref{table:Temperatur}). Allerdings bleibt die Lumineszenz in der kalten Lösung länger erhalten als in der Lösung bei Raumtemperatur. Genaue Werte dessen wurden jedoch nicht gemessen. Somit handelt es sich bei der Dauer der Lumineszenz um eine weitgehend subjektive Beobachtung.

\begin{dsatable}
 \caption{Messung der Intensität der Lumineszenz der Gemische bei 10 $^\circ$C und bei 20 $^\circ$C.}
 \centering
 \begin{tabular}{crr} 
  \toprule
  Zeit      &  10 $^\circ$C  &  20 $^\circ$C \\
  \midrule
   2s		& 10,09 	& 15,54 \\
   4s		& 8,55		& 9,76	\\
   6s		&  			& 6,76	\\
   8s		& 			& 4,82	\\
  \bottomrule
 \end{tabular}
 \label{table:Temperatur}
\end{dsatable}

\subsubsection{Auswertung}
Durch das Absenken der Temperatur wird die Bewegung der Teilchen verlangsamt. Somit treffen die Edukte seltener aufeinander. Dies verringert die Reaktionsgeschwindigkeit, sodass das Gemisch schwächer Licht emittiert. Außerdem dauert es länger, bis sich ein Reaktionsgleichgewicht einstellt, sodass die Lumineszenz länger erhalten bleibt. 
