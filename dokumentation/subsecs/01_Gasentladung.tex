\section{Gasentladung}
\authors{Gala Gottschalg, David Bürg}

In einer Gasentladungslampe wird durch Anlegen einer Spannung ein elektrisches
Feld ausgebildet.
\cite{Gerthsen}
\cite{Light_sources}
\cite{CIUZ:Lichtquellen}
In diesem elektrischen Feld werden Elektronen beschleunigt, die durch Stöße mit dem Füllgas zu dessen Anregung oder Ionisation führen. Ein Molekül oder Atom kann sich in verschiedenen energetischen Zuständen befinden. Der erste beziehungsweise energetisch niedrigste Zustand ist der Grundzustand. Bei einer Anregung wird ein Molekül oder Atom durch Energiezufuhr von außen auf ein höheres Energieniveau angehoben. Anschließend fällt das Teilchen in den Grundzustand zurück und überträgt die überschüssige Energie auf ein Photon. Im Falle einer Ionisation beziehungsweise einer Anregung ins Kontinuum kommt es nach einer gewissen Zeit wieder zur Rekombination der durch Ionisation entstandenen freien Elektronen und Kationen. Auch hier wird die überschüssige Energie auf ein Photon übertragen.
Die Wellenlänge des emittierten Photons beziehungsweise die Farbe des emittierten Lichts ist direkt abhängig vom angeregten Stoff. Das liegt daran, dass die Energiedifferenzen der unterschiedlichen Energieniveaus stoffspezifisch sind. 
Zudem ist das Spektrum des emittierten Lichts abhängig vom Druck des Füllgases, da ein höherer Druck zu einer Verbreitung der Linien im Emissionsspektrum führt. Diese Linien beziehen sich auf die Bereiche im elektromagnetischen Spektrum, in denen das Licht emittiert wird. Daraus folgt, dass die Linien im Emissionsspektrum der Hochdruckdampflampen breiter sind als bei einer Niederdruckdampflampe.

\begin{dsafigure}
 \centering
 \includegraphics[width=\columnwidth]{Xenon.png}
 \caption{Linienspektrum der Xenon-Hochdruckgasentladungslampe
         \cite{XenonArcLamp_wikipedia}.
         }
 \label{fig:Linienspektrum}
\end{dsafigure}

Das Emissionsspektrum der Xenon-Hochdruckgasentladungslampe in Abbildung \ref{fig:Linienspektrum} zeigt, dass die Linien bei kleinen Wellenlängen so breit sind, dass man diese nicht mehr voneinander unterscheiden kann. Die Emission verläuft daher kontinuierlich. Im Gegensatz dazu kann man im Bereich von 800--1000 nm einzelne Linien noch deutlich erkennen.
Weiterhin ist die Farbe des von der Lampe ausgehenden Lichts von einem Leuchtstoff in der Röhrenwand abhängig. Dieser Leuchtstoff wandelt kurzwellige UV-Strahlung in langwelliges, sichtbares Licht um.
Dies ist nötig, da UV-Strahlung schädlich ist und wir diese nicht sehen können.
