\section{Phosphoreszenz}
\authors{Selin Güler, Joes Biburger}

Mit Phosphoreszenz wird die Emission von Licht bezeichnet, die auf einen Übergang von einem angeregten Triplett- in einen Singulett-Grundzustand zurückzuführen ist. 

Der Phosphoreszenzvorgang beginnt mit der Absorption eines Photons mit spezifischer Wellenlänge durch ein Molekül. Letzteres wird vom Grundzustand in einen angeregten Singulettzustand $(S_1)$, in dem alle Elektronen mit entgegengesetztem Spin gepaart sind, angeregt. Nun erfolgt der als Intersystem Crossing bezeichnete Übergang in einen angeregten Triplettzustand $(T_1)$. In diesem gibt es zwei ungepaarte Elektronen mit gleichem Spin. Der angeregte Triplettzustand $(T_1)$ ist energieärmer und somit stabiler als der angeregte Singulettzustand $(S_1)$. Aufgrund der Tatsache, dass die für den Übergang in den Grundzustand $(S_0)$ notwendige Spinumkehr spinverboten und somit unwahrscheinlich ist, können mehrere Stunden vergehen, bevor das Molekül in den Grundzustand zurückfällt. Bei einem solchen Übergang wird Energie in Form von Licht abgegeben. Das emittierte Licht ist immer langwelliger als das bei der Anregung aufgenommene. Das rührt daher, dass bei der Phosphoreszenz, ähnlich wie bei der Fluoreszenz, verschiedene Schwingungszustände innerhalb des Grund- $(S_0)$ sowie angeregten Triplettzustandes $(T_1)$ existieren. Das Molekül gibt beim Übergehen in den energieärmsten Schwingungszustand Energie strahlungsfrei an weitere Translations-, Rotations-, Schwingungsmoden ab, weshalb nur ein Teil der Energie des absorbierten Lichtes in Form von Licht wieder emittiert wird. 
\cite{Phosphoreszenz1, Phosphoreszenz2, Phosphoreszenz3}
