\section{Aus der Kursbeschreibung}
\authors{Elke Fasshauer KL, Mats Simmermacher KL}

Woher kommt die Farbe chemischer Verbindungen? Lässt sich vorhersagen,
welche Farbe eine bestimmte Verbindung hat? Und was ist eigentlich Farbe?
Diesen und mehr Fragen sind wir im Kurs anhand verschiedener blauer
Farbstoffe und weiterer blauer Verbindungen nachgegangen.

Dazu wurden die physikalischen Grundlagen der Chemie betrachtet.
Alle Moleküle sind aus sehr wenigen Bausteinen aufgebaut: positiv geladenen
Atomkernen (von denen es in der Natur nur 92 verschiedene gibt) und negativ
geladenen Elektronen. Das Verhalten solcher kleiner geladener Teilchen wird
durch die
Gesetze der Quantenmechanik bestimmt, die in der Physik eingehend erforscht
wurden. Dadurch ist es im Prinzip möglich, die Chemie auf der Grundlage dieser
physikalischen Gesetze zu erklären. Auch die Farbigkeit von chemischen
Verbindungen beruht daher
letztlich auf der Quantenmechanik und so müssten sie sich auf dieser
physikalischen Basis verstehen lassen. In diesem Kurs haben wir betrachtet,
in wieweit ein solches Verständnis möglich ist.

Im Mittelpunkt der Kursarbeit stehen Referate der Teilnehmenden. Im ersten Teil
wurden die Grundlagen von Farbigkeit und einige Farbstoffe exemplarisch
behandelt.
Diese wurden dann im Kurs genauer unter die Lupe genommen.
Dafür wurden im zweiten Teil die grundlegenden Theorien und Modelle erarbeitet,
die benötigt werden, um Atome und Moleküle sowie ihre spektroskopischen
Eigenschaften zu verstehen. Hierfür wurden im Kurs auch die Grundlagen der
Quantenmechanik erarbeitet.

Gleichzeitig haben die Teilnehmenden in Computerprojekten die Farbigkeit von
Molekülen selbst theoretisch untersucht. Dabei wurden moderne Programme aus
der theoretischen Chemie verwendet, mit deren Hilfe sich Eigenschaften und
Reaktionen von Molekülen auf der Grundlage der Quantenmechanik berechnen
und vorhersagen lassen.
