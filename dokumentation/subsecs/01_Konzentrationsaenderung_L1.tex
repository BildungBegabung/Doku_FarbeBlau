\subsection{Konzentrationsveränderung von L1} 

Lynn Meeder, Selin Güler

\subsubsection{Versuchsbeschreibung}

Zunächst wird untersucht, wie sich eine Konzentrationsveränderung des Luminols und der Natronlauge (G1) auf den Reaktionsverlauf auswirkt. G2 bleibt während beider Varianten des Experimentes konstant. 

\subsubsection{Beobachtung}

Bei hoher Konzentration des Luminols und der Natronlauge ($G1^+$) ist eine sehr schnelle Reaktion zu beobachten. Auch ist direkt nach dem Zusammengeben von $G1^+$ mit der Lösung von G2 ist ein sehr starkes Leuchten zu sehen, das jedoch schnell an Intensität verliert. Die Intensität des emittierten Lichtes ist bereits nach 4 Sekunden so gering, dass die verwendete Kameraeinstellung aufgrund der schwachen Lumineszenz keine Bildaufnahme zulässt. Setzt man die Grundlösung aus der Stammlösung G1 mit halber Konzentration an, ergibt sich eine Grundlösung ($G1^-$), die bei der Reaktion mit G2 so wenig Licht emittiert, dass die Kamera bereits zu Beginn der Reaktion nicht auslöst.

\subsubsection{Auswertung}

Zusammenfassend kann gesagt werden, dass das Erhöhen der Konzentration von G1 die Reaktionsgeschwindigkeit ansteigen lässt und ein intensiveres Licht emittiert wird. Wenn die Konzentration von G1 herabgesetzt wird, tritt der gegenteilige Effekt ein.