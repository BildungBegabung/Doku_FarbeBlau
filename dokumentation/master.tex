\documentclass{newlayout}
%Bitte hier den enstprechenden Ort einsetzen z.B. Braunschweig und die Akademienummer
\Akademie{Ort}{2014}{1}

\usepackage[ngerman,english]{babel}
\usepackage{misc}
\usepackage{multicol}
\usepackage{booktabs}

\usepackage{color}% für Farben im allgemeinen
\usepackage{colortbl}

\usepackage{url}
\usepackage{breakurl}

\usepackage{units}

%\usepackage{amsmath}%wird automatisch durch newlayout.cls geladen
\usepackage{amsfonts}

%%%%%Mathe-Definitionen
\newtheorem{Def}{Definition}
\newtheorem{Sat}{Satz}
\newtheorem{Bew}{Beweis}

%%%%Ende Mathe-Definitionen

\begin{document}

 %   \input{titel}
 \setcounter{page}{3}

\setcounter{tocdepth}{1}
 \tableofcontents

   \setcounter{secnumdepth}{1}


\setcounter{page}{7}
\setcounter{chapter}{0}

%Angabe, bis zu welcher Stufe die sections im Text nummeriert werden sollen.
      \settocdepth{2}

\graphicspath{ {./pics/} }


\course{1}{Die Farbe Blau}%%% 
\begin{coursetitle}
  \centerline{Die Farbe Blau} 
  \bigskip
  %\Large \centerline{Kursuntertitel eingeben}
  \bigskip
 %\includegraphics[width=.9\textwidth]{kurslogo.png}
 \label{fig:meinbild}
  \bigskip
\end{coursetitle}


\section{Gasentladung}
\authors{Gala Gottschalg, David Bürg}

In einer Gasentladungslampe wird durch Anlegen einer Spannung ein elektrisches
Feld ausgebildet.
\cite{Gerthsen}
\cite{Light_sources}
\cite{CIUZ:Lichtquellen}
In diesem elektrischen Feld werden Elektronen beschleunigt, die durch Stöße mit dem Füllgas zu dessen Anregung oder Ionisation führen. Ein Molekül oder Atom kann sich in verschiedenen energetischen Zuständen befinden. Der erste beziehungsweise energetisch niedrigste Zustand ist der Grundzustand. Bei einer Anregung wird ein Molekül oder Atom durch Energiezufuhr von außen auf ein höheres Energieniveau angehoben. Anschließend fällt das Teilchen in den Grundzustand zurück und überträgt die überschüssige Energie auf ein Photon. Im Falle einer Ionisation beziehungsweise einer Anregung ins Kontinuum kommt es nach einer gewissen Zeit wieder zur Rekombination der durch Ionisation entstandenen freien Elektronen und Kationen. Auch hier wird die überschüssige Energie auf ein Photon übertragen.
Die Wellenlänge des emittierten Photons beziehungsweise die Farbe des emittierten Lichts ist direkt abhängig vom angeregten Stoff. Das liegt daran, dass die Energiedifferenzen der unterschiedlichen Energieniveaus stoffspezifisch sind. 
Zudem ist das Spektrum des emittierten Lichts abhängig vom Druck des Füllgases, da ein höherer Druck zu einer Verbreitung der Linien im Emissionsspektrum führt. Diese Linien beziehen sich auf die Bereiche im elektromagnetischen Spektrum, in denen das Licht emittiert wird. Daraus folgt, dass die Linien im Emissionsspektrum der Hochdruckdampflampen breiter sind als bei einer Niederdruckdampflampe.

\begin{dsafigure}
 \centering
 \includegraphics[width=\columnwidth]{Xenon.png}
 \caption{Linienspektrum der Xenon-Hochdruckgasentladungslampe
         \cite{XenonArcLamp_wikipedia}.
         }
 \label{fig:Linienspektrum}
\end{dsafigure}

Das Emissionsspektrum der Xenon-Hochdruckgasentladungslampe in Abbildung \ref{fig:Linienspektrum} zeigt, dass die Linien bei kleinen Wellenlängen so breit sind, dass man diese nicht mehr voneinander unterscheiden kann. Die Emission verläuft daher kontinuierlich. Im Gegensatz dazu kann man im Bereich von 800--1000 nm einzelne Linien noch deutlich erkennen.
Weiterhin ist die Farbe des von der Lampe ausgehenden Lichts von einem Leuchtstoff in der Röhrenwand abhängig. Dieser Leuchtstoff wandelt kurzwellige UV-Strahlung in langwelliges, sichtbares Licht um.
Dies ist nötig, da UV-Strahlung schädlich ist und wir diese nicht sehen können.

\section{Hückeltheorie}

Autoren: David Bürg, Isabelle Schulte-Herbrüggen

Die Hückeltheorie wurde in den 1930er Jahren von Erick Hückel entwickelt. \cite{Reinhold} Sie erlaubt eine einfache Beschreibung von konjugierten Doppelbindungssystemen. Die Methode arbeitet mit vielen Näherungen. Daher ist sie zwar ungenau, aber einfach anzuwenden. Sie baut auf der Näherung von $\pi$-Molekülorbitalen und deren Energien über Linearkombinationen von p-Atomorbitalen auf. Die Linearkombinationen können mit:

\begin{align}
 \psi_{i} = \sum \limits_{k=1}^n c_{ik} \chi_k 
\end{align}

beschrieben werden. Der Koeffizient $c_{ik}$ gibt hierbei an, wie  stark die einzelnen Atomorbitale $\chi_k$ am Molekülorbital $\psi_i$ beteiligt sind. Die Koeffizienten $c_{ik}$ lassen sich über die Summen:

\begin{align}\label{eq:hueckel}
  \sum \limits_{k=1}^n (H_{jk}-\epsilon_i S_{jk}) c_{ik} = 0
\end{align}

bestimmen. $H_{jk}$ ist das Hamiltonmatrixelement, $S_{jk}$ das Überlappmatrixselement und $\epsilon_i$ ist die Energie des Molekülorbitals. Mit Gleichung (\ref{eq:hueckel}) kann nun die Hückelmatrix formuliert werden. Dabei wird angenommen, dass die Hamiltonmatrixelemente aller Kohlenstoffatome gleich sind, sowie dass die Wechselwirkungen aller nächsten Nachbarn gleich sind. Zusätzlich vereinfacht man die Beschreibung noch weiter, indem die Wechselwirkungen zwischen nicht-nächsten Nachbarn komplett vernachlässigt werden. Aufgrund dieser Vereinfachungen sind die Ergebnisse, die man durch Anwenden der Hückeltheorie erhält, grobe Näherungen.


Indem man fordert, dass die Determinante der Hückelmatrix verschwindet, erhält man einen Satz gekoppelter Gleichungen, deren Lösung die Koeffizienten $c_{ik}$ und die Orbitalenergien $\epsilon_i$ liefert. So kann nun die Energiedifferenz zwischen HOMO und LUMO mit:

\begin{align}
  \Delta E = \epsilon_{LUMO} - \epsilon_{HOMO}
\end{align}

berechnet werden. Diese Energiedifferenz erlaubt eine grobe Abschätzung der ersten Anregungsenergie des Moleküls.

Beschreibt man das 1,3-Butadien-Molekül mit der Hückelthoerie, erhält man die in Abbildung  \ref{fig:Hueckel_Butadiene} dargestellten Molekülorbitale.

\begin{dsafigure}
 \centering
 \includegraphics[width=8cm]{pics/Hueckel_Butadiene.png}
 \caption{Aufsicht auf ein 1,3-Butadienmolekül. Schematische Darstellung der Hückelmolekülorbitale von 1,3-Butadien.}
 \label{fig:Hueckel_Butadiene}
\end{dsafigure}

Die Radien der Kreise entsprechen den Koeffizienten $c_{ik}$, während Weiß und Grau die Positivteile beziehungsweise Negativteile beschreiben. Das erste Orbital ist das energetisch günstigste Molekülorbital von Butadien, da es nur bindende Wechselwirkungen aufweist und keine Knotenebene besitzt; es ist ein bindendes Orbital. Das zweite Orbital hingegen weist zwei bindende Wechselwirkungen und eine Knotenebene auf. Es handelt sich auch hier um ein bindendes Orbital. Beim dritten un vierten Orbital handelt es sich um antibindende Orbitale. Das dritte Orbital zeigt zwei Knotenebenen und eine bindende Wechselwirkung und das vierte Orbital weist drei Knotenebenen und keine bindenden Wechselwirkungen auf. Sie haben also mehr Knotenebenen als bindende Wechselwirkungen. Die Energie vier Molekülorbitale nimmt in aufsteigender Reihenfolge zu.



\section*{Literaturverzeichnis}
\bibliographystyle{jcpsty_deutsch}
%\bibliographystyle{unsrtdin}
\bibliography{lit2}

\end{document}
