\documentclass{newlayout}
%Bitte hier den enstprechenden Ort einsetzen z.B. Braunschweig und die Akademienummer
\Akademie{Ort}{2014}{1}

\usepackage[ngerman,english]{babel}
\usepackage{misc}
\usepackage{multicol}
\usepackage{booktabs}

\usepackage{color}% für Farben im allgemeinen
\usepackage{colortbl}

\usepackage{url}
\usepackage{breakurl}

\usepackage{units}

%\usepackage{amsmath}%wird automatisch durch newlayout.cls geladen
\usepackage{amsfonts}

%%%%%Mathe-Definitionen
\newtheorem{Def}{Definition}
\newtheorem{Sat}{Satz}
\newtheorem{Bew}{Beweis}

%%%%Ende Mathe-Definitionen

\begin{document}

 %   \input{titel}
 \setcounter{page}{3}

\setcounter{tocdepth}{1}
 \tableofcontents

   \setcounter{secnumdepth}{1}


\setcounter{page}{7}
\setcounter{chapter}{0}

%Angabe, bis zu welcher Stufe die sections im Text nummeriert werden sollen.
      \settocdepth{2}

\graphicspath{ {./pics/} }


\course{1}{Die Farbe Blau}%%% 
\begin{coursetitle}
  \centerline{Die Farbe Blau} 
  \bigskip
  %\Large \centerline{Kursuntertitel eingeben}
  \bigskip
 %\includegraphics[width=.9\textwidth]{kurslogo.png}
 \label{fig:meinbild}
  \bigskip
\end{coursetitle}


\section{Gasentladung}



\section*{Literaturverzeichnis}
\bibliographystyle{jcpsty_deutsch}
%\bibliographystyle{unsrtdin}
\bibliography{lit2}

\end{document}
